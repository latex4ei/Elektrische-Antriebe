% .:: Laden der LaTeX4EI Formelsammlungsvorlage
\documentclass[fs, footer]{latex4ei}

% Dokumentbeginn
% ======================================================================
\begin{document}

% Aufteilung in Spalten
\begin{multicols*}{4}
\fstitle{Elektrische\\ Antriebe}

% -------------------------------------------
% | 		Elektrische Antriebe			|
% ~~~~~~~~~~~~~~~~~~~~~~~~~~~~~~~~~~~~~~~~~~~
% SECTION ====================================================================================
\section{Einleitung}
% ============================================================================================

Konzentriertes Bauelement: Falls es durch äußere Gleichungen bestimmt ist und keine Energiespeicher im Inneren hat.
Elektrische Antriebe sind keine konzentrierten Bauelemente, da der innere Ablauf entscheidend ist.

Homogenes Feld: In jedem Punkt gleiche Richtung, gleiche Wellenlänge und Stärke. \\
Elektrische Maschine: Wandler Strom$\leftrightarrow$Bewegung; Motor bzw. Generator.\\

Begrenzungen in Drehzahl, Strom, 

Vorraussetzung für Regelungskochrezepte: Eine dominante Zeitkonstante! 



	\subsection{Servo-Antriebe}
	Servoantrieb: Antrieb, der Kommandos folgen kann, z.B. Position anfahren.\\
	Kette: Prozessor $\xrightarrow[{}^{\longleftarrow}]{\text{Regelung}}$ Energiewandler $\longrightarrow$ Motor\\

	Arten: Stellantriebe (Positionierung), funktionale Antriebe (Energie umsetzen)

	Moderne Servos: permanent erregte Synchronmaschine mit digitaler Regelung.



	\subsection{Mechanische Kopplung}
	\begin{itemize}
		\item Zahnstange/Ritzel
		\item Kugelrollspindel
		\item Kette (Metall), Zahnriemen (Kunststoff)
		\item Getriebe, Zahnräder
	\end{itemize}

% Zitat: Ein Ingenieur ist das optimale Verhältnis aus Kreativität und Faulheit. Faulheit ist nicht ehrenwehrtes man muss es nur kultivieren.

% Knorrbremse


% Ende der Spalten
\end{multicols*}

% Dokumentende
% ======================================================================
\end{document}

% ToDos:

